%
% API Documentation for API Documentation
% Module HttpServer
%
% Generated by epydoc 3.0.1
% [Wed Jan 19 23:31:59 2011]
%

%%%%%%%%%%%%%%%%%%%%%%%%%%%%%%%%%%%%%%%%%%%%%%%%%%%%%%%%%%%%%%%%%%%%%%%%%%%
%%                          Module Description                           %%
%%%%%%%%%%%%%%%%%%%%%%%%%%%%%%%%%%%%%%%%%%%%%%%%%%%%%%%%%%%%%%%%%%%%%%%%%%%

    \index{HttpServer \textit{(module)}|(}
\section{Module HttpServer}

    \label{HttpServer}
The HTTP server module. Takes care of serving user requests and issuing 
data requests to database

\begin{itemize}
\setlength{\parskip}{0.6ex}
  \item initializes all other back-end modules, and closes them afterwards

  \item serves forever until terminated by the user

\end{itemize}


%%%%%%%%%%%%%%%%%%%%%%%%%%%%%%%%%%%%%%%%%%%%%%%%%%%%%%%%%%%%%%%%%%%%%%%%%%%
%%                               Variables                               %%
%%%%%%%%%%%%%%%%%%%%%%%%%%%%%%%%%%%%%%%%%%%%%%%%%%%%%%%%%%%%%%%%%%%%%%%%%%%

  \subsection{Variables}

    \vspace{-1cm}
\hspace{\varindent}\begin{longtable}{|p{\varnamewidth}|p{\vardescrwidth}|l}
\cline{1-2}
\cline{1-2} \centering \textbf{Name} & \centering \textbf{Description}& \\
\cline{1-2}
\endhead\cline{1-2}\multicolumn{3}{r}{\small\textit{continued on next page}}\\\endfoot\cline{1-2}
\endlastfoot\raggedright P\-O\-R\-T\- & \raggedright The default Port number which the server uses to listen to user 
          requests

\textbf{Value:} 
{\tt 6789}&\\
\cline{1-2}
\raggedright s\-e\-r\-i\-a\-l\- & \raggedright Instance of the SComms module

\textbf{Value:} 
{\tt None}&\\
\cline{1-2}
\end{longtable}


%%%%%%%%%%%%%%%%%%%%%%%%%%%%%%%%%%%%%%%%%%%%%%%%%%%%%%%%%%%%%%%%%%%%%%%%%%%
%%                           Class Description                           %%
%%%%%%%%%%%%%%%%%%%%%%%%%%%%%%%%%%%%%%%%%%%%%%%%%%%%%%%%%%%%%%%%%%%%%%%%%%%

    \index{HttpServer \textit{(module)}!HttpServer.Handler \textit{(class)}|(}
\subsection{Class Handler}

    \label{HttpServer:Handler}
\begin{tabular}{cccccc}
% Line for http.server.SimpleHTTPRequestHandler, linespec=[False]
\multicolumn{2}{r}{\settowidth{\BCL}{http.server.SimpleHTTPRequestHandler}\multirow{2}{\BCL}{http.server.SimpleHTTPRequestHandler}}
&&
  \\\cline{3-3}
  &&\multicolumn{1}{c|}{}
&&
  \\
&&\multicolumn{2}{l}{\textbf{HttpServer.Handler}}
\end{tabular}

The Handler class, used to handle user requests.


%%%%%%%%%%%%%%%%%%%%%%%%%%%%%%%%%%%%%%%%%%%%%%%%%%%%%%%%%%%%%%%%%%%%%%%%%%%
%%                                Methods                                %%
%%%%%%%%%%%%%%%%%%%%%%%%%%%%%%%%%%%%%%%%%%%%%%%%%%%%%%%%%%%%%%%%%%%%%%%%%%%

  \subsubsection{Methods}

    \label{HttpServer:Handler:get_params}
    \index{HttpServer \textit{(module)}!HttpServer.Handler \textit{(class)}!HttpServer.Handler.get\_params \textit{(method)}}

    \vspace{0.5ex}

\hspace{.8\funcindent}\begin{boxedminipage}{\funcwidth}

    \raggedright \textbf{get\_params}(\textit{self})

    \vspace{-1.5ex}

    \rule{\textwidth}{0.5\fboxrule}
\setlength{\parskip}{2ex}
    Parses the URL given by the web-based frontend, and extracts the 
    parameters

\setlength{\parskip}{1ex}
      \textbf{Return Value}
    \vspace{-1ex}

      \begin{quote}
      a list of parameters

      {\it (type=array)}

      \end{quote}

    \end{boxedminipage}

    \label{HttpServer:Handler:errorDoc}
    \index{HttpServer \textit{(module)}!HttpServer.Handler \textit{(class)}!HttpServer.Handler.errorDoc \textit{(method)}}

    \vspace{0.5ex}

\hspace{.8\funcindent}\begin{boxedminipage}{\funcwidth}

    \raggedright \textbf{errorDoc}(\textit{self}, \textit{errorText})

    \vspace{-1.5ex}

    \rule{\textwidth}{0.5\fboxrule}
\setlength{\parskip}{2ex}
    Creates a custom error document if the request is not recognised

\setlength{\parskip}{1ex}
      \textbf{Parameters}
      \vspace{-1ex}

      \begin{quote}
        \begin{Ventry}{xxxxxxxxx}

          \item[errorText]

          specifies the text to be printed on the error page

            {\it (type=string)}

        \end{Ventry}

      \end{quote}

    \end{boxedminipage}

    \label{HttpServer:Handler:do_GET}
    \index{HttpServer \textit{(module)}!HttpServer.Handler \textit{(class)}!HttpServer.Handler.do\_GET \textit{(method)}}

    \vspace{0.5ex}

\hspace{.8\funcindent}\begin{boxedminipage}{\funcwidth}

    \raggedright \textbf{do\_GET}(\textit{self})

    \vspace{-1.5ex}

    \rule{\textwidth}{0.5\fboxrule}
\setlength{\parskip}{2ex}
    Custom handler for user requests

    All URL requests start with the '/data' delimeter

\setlength{\parskip}{1ex}
    \end{boxedminipage}

    \index{HttpServer \textit{(module)}!HttpServer.Handler \textit{(class)}|)}
    \index{HttpServer \textit{(module)}|)}
