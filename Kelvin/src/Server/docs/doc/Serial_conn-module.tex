%
% API Documentation for API Documentation
% Module Serial_conn
%
% Generated by epydoc 3.0.1
% [Wed Jan 19 23:32:00 2011]
%

%%%%%%%%%%%%%%%%%%%%%%%%%%%%%%%%%%%%%%%%%%%%%%%%%%%%%%%%%%%%%%%%%%%%%%%%%%%
%%                          Module Description                           %%
%%%%%%%%%%%%%%%%%%%%%%%%%%%%%%%%%%%%%%%%%%%%%%%%%%%%%%%%%%%%%%%%%%%%%%%%%%%

    \index{Serial\_conn \textit{(module)}|(}
\section{Module Serial\_conn}

    \label{Serial_conn}
This is the main Serial Communications file.

Establishes serial communication with the Base Station and initializes the 
Sensor\_thread.py, which takes care of retrieving sensor data and storing 
it to database.

The module has preconfigured values for baudrate and timeout, but can be 
explicitly specified by the user.

The module defines the data transmission mechanism using base64 
implementation


%%%%%%%%%%%%%%%%%%%%%%%%%%%%%%%%%%%%%%%%%%%%%%%%%%%%%%%%%%%%%%%%%%%%%%%%%%%
%%                           Class Description                           %%
%%%%%%%%%%%%%%%%%%%%%%%%%%%%%%%%%%%%%%%%%%%%%%%%%%%%%%%%%%%%%%%%%%%%%%%%%%%

    \index{Serial\_conn \textit{(module)}!Serial\_conn.Serial\_conn \textit{(class)}|(}
\subsection{Class Serial\_conn}

    \label{Serial_conn:Serial_conn}
Serial communication class. Provides connection and message transmission 
functionalities.


%%%%%%%%%%%%%%%%%%%%%%%%%%%%%%%%%%%%%%%%%%%%%%%%%%%%%%%%%%%%%%%%%%%%%%%%%%%
%%                                Methods                                %%
%%%%%%%%%%%%%%%%%%%%%%%%%%%%%%%%%%%%%%%%%%%%%%%%%%%%%%%%%%%%%%%%%%%%%%%%%%%

  \subsubsection{Methods}

    \label{Serial_conn:Serial_conn:__init__}
    \index{Serial\_conn \textit{(module)}!Serial\_conn.Serial\_conn \textit{(class)}!Serial\_conn.Serial\_conn.\_\_init\_\_ \textit{(method)}}

    \vspace{0.5ex}

\hspace{.8\funcindent}\begin{boxedminipage}{\funcwidth}

    \raggedright \textbf{\_\_init\_\_}(\textit{self})

    \vspace{-1.5ex}

    \rule{\textwidth}{0.5\fboxrule}
\setlength{\parskip}{2ex}
    Initializes the SComms module.

\setlength{\parskip}{1ex}
    \end{boxedminipage}

    \label{Serial_conn:Serial_conn:get_port}
    \index{Serial\_conn \textit{(module)}!Serial\_conn.Serial\_conn \textit{(class)}!Serial\_conn.Serial\_conn.get\_port \textit{(method)}}

    \vspace{0.5ex}

\hspace{.8\funcindent}\begin{boxedminipage}{\funcwidth}

    \raggedright \textbf{get\_port}(\textit{self})

    \vspace{-1.5ex}

    \rule{\textwidth}{0.5\fboxrule}
\setlength{\parskip}{2ex}
    Specified the port number to be opened. Scans all available ports, and 
    prompts the user to select a port number to be opened

\setlength{\parskip}{1ex}
      \textbf{Return Value}
    \vspace{-1ex}

      \begin{quote}
      the port number to be opened

      {\it (type=number)}

      \end{quote}

    \end{boxedminipage}

    \label{Serial_conn:Serial_conn:get_baud}
    \index{Serial\_conn \textit{(module)}!Serial\_conn.Serial\_conn \textit{(class)}!Serial\_conn.Serial\_conn.get\_baud \textit{(method)}}

    \vspace{0.5ex}

\hspace{.8\funcindent}\begin{boxedminipage}{\funcwidth}

    \raggedright \textbf{get\_baud}(\textit{self}, \textit{msg}={\tt "Please enter the baudrate of the connection, or press EN\texttt{...}})

    \vspace{-1.5ex}

    \rule{\textwidth}{0.5\fboxrule}
\setlength{\parskip}{2ex}
    Prompts the user for a baudrate value. If none selected, a default 
    value is chosen.

\setlength{\parskip}{1ex}
      \textbf{Parameters}
      \vspace{-1ex}

      \begin{quote}
        \begin{Ventry}{xxx}

          \item[msg]

          the prompt message to be displayed to the user.

            {\it (type=string)}

        \end{Ventry}

      \end{quote}

      \textbf{Return Value}
    \vspace{-1ex}

      \begin{quote}
      the baudrate value

      {\it (type=number)}

      \end{quote}

    \end{boxedminipage}

    \label{Serial_conn:Serial_conn:get_timeout}
    \index{Serial\_conn \textit{(module)}!Serial\_conn.Serial\_conn \textit{(class)}!Serial\_conn.Serial\_conn.get\_timeout \textit{(method)}}

    \vspace{0.5ex}

\hspace{.8\funcindent}\begin{boxedminipage}{\funcwidth}

    \raggedright \textbf{get\_timeout}(\textit{self}, \textit{msg}={\tt "Please enter the connection timeout (in seconds), or pre\texttt{...}})

    \vspace{-1.5ex}

    \rule{\textwidth}{0.5\fboxrule}
\setlength{\parskip}{2ex}
    Prompts the user for a timeout value. If none selected, a default value
    is chosen.

\setlength{\parskip}{1ex}
      \textbf{Parameters}
      \vspace{-1ex}

      \begin{quote}
        \begin{Ventry}{xxx}

          \item[msg]

          the prompt message to be displayed to the user.

            {\it (type=string)}

        \end{Ventry}

      \end{quote}

      \textbf{Return Value}
    \vspace{-1ex}

      \begin{quote}
      the timeout value

      {\it (type=number)}

      \end{quote}

    \end{boxedminipage}

    \label{Serial_conn:Serial_conn:close}
    \index{Serial\_conn \textit{(module)}!Serial\_conn.Serial\_conn \textit{(class)}!Serial\_conn.Serial\_conn.close \textit{(method)}}

    \vspace{0.5ex}

\hspace{.8\funcindent}\begin{boxedminipage}{\funcwidth}

    \raggedright \textbf{close}(\textit{self}, \textit{status}={\tt 0})

    \vspace{-1.5ex}

    \rule{\textwidth}{0.5\fboxrule}
\setlength{\parskip}{2ex}
    Terminates the serial communication and closes the the sensor\_thread 
    instance

\setlength{\parskip}{1ex}
      \textbf{Parameters}
      \vspace{-1ex}

      \begin{quote}
        \begin{Ventry}{xxxxxx}

          \item[status]

          the close status value

            {\it (type=number)}

        \end{Ventry}

      \end{quote}

    \end{boxedminipage}

    \label{Serial_conn:Serial_conn:start}
    \index{Serial\_conn \textit{(module)}!Serial\_conn.Serial\_conn \textit{(class)}!Serial\_conn.Serial\_conn.start \textit{(method)}}

    \vspace{0.5ex}

\hspace{.8\funcindent}\begin{boxedminipage}{\funcwidth}

    \raggedright \textbf{start}(\textit{self})

    \vspace{-1.5ex}

    \rule{\textwidth}{0.5\fboxrule}
\setlength{\parskip}{2ex}
    Starts the SComms module. Prompts the user for a port number, baudrate 
    and timeout values, and connects to the Base Station using these 
    values.

\setlength{\parskip}{1ex}
    \end{boxedminipage}

    \label{Serial_conn:Serial_conn:base_st_connect}
    \index{Serial\_conn \textit{(module)}!Serial\_conn.Serial\_conn \textit{(class)}!Serial\_conn.Serial\_conn.base\_st\_connect \textit{(method)}}

    \vspace{0.5ex}

\hspace{.8\funcindent}\begin{boxedminipage}{\funcwidth}

    \raggedright \textbf{base\_st\_connect}(\textit{self})

    \vspace{-1.5ex}

    \rule{\textwidth}{0.5\fboxrule}
\setlength{\parskip}{2ex}
    Sends a confirmation message to Base Station to acknowledge connection.
    If it receives and ACK from Base Station, starts up the Sensor\_thread.

\setlength{\parskip}{1ex}
    \end{boxedminipage}

    \label{Serial_conn:Serial_conn:connect}
    \index{Serial\_conn \textit{(module)}!Serial\_conn.Serial\_conn \textit{(class)}!Serial\_conn.Serial\_conn.connect \textit{(method)}}

    \vspace{0.5ex}

\hspace{.8\funcindent}\begin{boxedminipage}{\funcwidth}

    \raggedright \textbf{connect}(\textit{self}, \textit{port}, \textit{baud}={\tt 38400}, \textit{\_timeout}={\tt 1})

    \vspace{-1.5ex}

    \rule{\textwidth}{0.5\fboxrule}
\setlength{\parskip}{2ex}
    Tries to connected to specified port, using given baudrate and timeout 
    values

\setlength{\parskip}{1ex}
      \textbf{Parameters}
      \vspace{-1ex}

      \begin{quote}
        \begin{Ventry}{xxxxxxxx}

          \item[port]

          the port to be opened

            {\it (type=number)}

          \item[baud]

          the specified baudrate

            {\it (type=number)}

          \item[\_timeout]

          the specified timeout value

            {\it (type=number)}

        \end{Ventry}

      \end{quote}

    \end{boxedminipage}

    \label{Serial_conn:Serial_conn:scan_for_ports}
    \index{Serial\_conn \textit{(module)}!Serial\_conn.Serial\_conn \textit{(class)}!Serial\_conn.Serial\_conn.scan\_for\_ports \textit{(method)}}

    \vspace{0.5ex}

\hspace{.8\funcindent}\begin{boxedminipage}{\funcwidth}

    \raggedright \textbf{scan\_for\_ports}(\textit{self})

    \vspace{-1.5ex}

    \rule{\textwidth}{0.5\fboxrule}
\setlength{\parskip}{2ex}
    Scans the system for open ports

\setlength{\parskip}{1ex}
      \textbf{Return Value}
    \vspace{-1ex}

      \begin{quote}
      an array of available ports

      {\it (type=array)}

      \end{quote}

    \end{boxedminipage}

    \label{Serial_conn:Serial_conn:read}
    \index{Serial\_conn \textit{(module)}!Serial\_conn.Serial\_conn \textit{(class)}!Serial\_conn.Serial\_conn.read \textit{(method)}}

    \vspace{0.5ex}

\hspace{.8\funcindent}\begin{boxedminipage}{\funcwidth}

    \raggedright \textbf{read}(\textit{self})

    \vspace{-1.5ex}

    \rule{\textwidth}{0.5\fboxrule}
\setlength{\parskip}{2ex}
    Calls \_process\_line to read incoming messages

\setlength{\parskip}{1ex}
    \end{boxedminipage}

    \label{Serial_conn:Serial_conn:write}
    \index{Serial\_conn \textit{(module)}!Serial\_conn.Serial\_conn \textit{(class)}!Serial\_conn.Serial\_conn.write \textit{(method)}}

    \vspace{0.5ex}

\hspace{.8\funcindent}\begin{boxedminipage}{\funcwidth}

    \raggedright \textbf{write}(\textit{self}, \textit{string})

    \vspace{-1.5ex}

    \rule{\textwidth}{0.5\fboxrule}
\setlength{\parskip}{2ex}
    Only responds in Base64 if a previously received line was Base64. Adds

\setlength{\parskip}{1ex}
    \end{boxedminipage}

    \label{Serial_conn:Serial_conn:__iter__}
    \index{Serial\_conn \textit{(module)}!Serial\_conn.Serial\_conn \textit{(class)}!Serial\_conn.Serial\_conn.\_\_iter\_\_ \textit{(method)}}

    \vspace{0.5ex}

\hspace{.8\funcindent}\begin{boxedminipage}{\funcwidth}

    \raggedright \textbf{\_\_iter\_\_}(\textit{self})

\setlength{\parskip}{2ex}
\setlength{\parskip}{1ex}
    \end{boxedminipage}

    \label{Serial_conn:Serial_conn:__next__}
    \index{Serial\_conn \textit{(module)}!Serial\_conn.Serial\_conn \textit{(class)}!Serial\_conn.Serial\_conn.\_\_next\_\_ \textit{(method)}}

    \vspace{0.5ex}

\hspace{.8\funcindent}\begin{boxedminipage}{\funcwidth}

    \raggedright \textbf{\_\_next\_\_}(\textit{self})

\setlength{\parskip}{2ex}
\setlength{\parskip}{1ex}
    \end{boxedminipage}

    \label{Serial_conn:Serial_conn:set_color}
    \index{Serial\_conn \textit{(module)}!Serial\_conn.Serial\_conn \textit{(class)}!Serial\_conn.Serial\_conn.set\_color \textit{(method)}}

    \vspace{0.5ex}

\hspace{.8\funcindent}\begin{boxedminipage}{\funcwidth}

    \raggedright \textbf{set\_color}(\textit{self}, \textit{col})

\setlength{\parskip}{2ex}
\setlength{\parskip}{1ex}
    \end{boxedminipage}

    \label{Serial_conn:Serial_conn:flash}
    \index{Serial\_conn \textit{(module)}!Serial\_conn.Serial\_conn \textit{(class)}!Serial\_conn.Serial\_conn.flash \textit{(method)}}

    \vspace{0.5ex}

\hspace{.8\funcindent}\begin{boxedminipage}{\funcwidth}

    \raggedright \textbf{flash}(\textit{self})

\setlength{\parskip}{2ex}
\setlength{\parskip}{1ex}
    \end{boxedminipage}

    \label{Serial_conn:Serial_conn:pause}
    \index{Serial\_conn \textit{(module)}!Serial\_conn.Serial\_conn \textit{(class)}!Serial\_conn.Serial\_conn.pause \textit{(method)}}

    \vspace{0.5ex}

\hspace{.8\funcindent}\begin{boxedminipage}{\funcwidth}

    \raggedright \textbf{pause}(\textit{self})

    \vspace{-1.5ex}

    \rule{\textwidth}{0.5\fboxrule}
\setlength{\parskip}{2ex}
    Pauses the sensor\_thread

    \begin{itemize}
    \setlength{\parskip}{0.6ex}
      \item Deprecated

    \end{itemize}

\setlength{\parskip}{1ex}
    \end{boxedminipage}

    \label{Serial_conn:Serial_conn:resume}
    \index{Serial\_conn \textit{(module)}!Serial\_conn.Serial\_conn \textit{(class)}!Serial\_conn.Serial\_conn.resume \textit{(method)}}

    \vspace{0.5ex}

\hspace{.8\funcindent}\begin{boxedminipage}{\funcwidth}

    \raggedright \textbf{resume}(\textit{self})

    \vspace{-1.5ex}

    \rule{\textwidth}{0.5\fboxrule}
\setlength{\parskip}{2ex}
    Resumes the sensor\_thread

    \begin{itemize}
    \setlength{\parskip}{0.6ex}
      \item Deprecated

    \end{itemize}

\setlength{\parskip}{1ex}
    \end{boxedminipage}

    \index{Serial\_conn \textit{(module)}!Serial\_conn.Serial\_conn \textit{(class)}|)}
    \index{Serial\_conn \textit{(module)}|)}
